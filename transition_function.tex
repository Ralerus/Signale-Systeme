\subsection{Die Übergangsfunktion}

Die Übergangsfunktion $h(t)$, auch Sprungantwort genannt, ist die Antwort im Zeitbereich auf die Heaviside-Funktion $\sigma(t)$: \[\sigma (t) = \begin{cases} 0 & \text{für t < 0} \\ 1 & \text{für t > 0} \end{cases}\] \newline
In Abbildung \ref{fig:aufbau} ist der Aufbau zu sehen. Abbildungen \ref{fig:oszilloskop} und \ref{fig:matlab100_sprungantwort} zeigen die Sprungantwort in Simulink als auch berechnet in Matlab. \\
Die Steigung der Asymptoten der Sprungantwort entspricht dem statischen Verstärkungsfaktor $K=4$.

\begin{figure}[H]
	\centering
	\includegraphics[width=0.8\textwidth]{{diagrams/aufbau.png}}
	\caption[Aufbau]{Aufbau in Simulink}
	\label{fig:aufbau}
\end{figure}

\begin{figure}[H]
	\centering
	\includegraphics[width=0.8\textwidth]{{diagrams/sprungantwort_simulink_100.png}}
	\caption[Oszilloskop]{Oszilloskop - Sprungantwort im Intervall 0 bis 100}
	\label{fig:oszilloskop}
\end{figure}

% \begin{figure}[H]
%	\centering
%	\includegraphics[width=0.8\textwidth]{{diagrams/sprungantwort_matlab.png}}
%	\caption[Oszilloskop]{Matlab - Sprungantwort}
%	\label{fig:matlab_sprungantwort}
% \end{figure}

\begin{figure}[H]
	\centering
	\includegraphics[width=0.8\textwidth]{{diagrams/sprungantwort_matlab_limit100.png}}
	\caption[Oszilloskop]{Matlab - Sprungantwort im Intervall 0 bis 100}
	\label{fig:matlab100_sprungantwort}
\end{figure}