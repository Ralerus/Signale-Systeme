\subsection{Zustandsraumdarstellung}

Die allgemeine Form der Zustandsraumdarstellung ist in den folgenden zwei Zeilen zu sehen.

\begin{eqnarray*}
	\dot x &=& Ax + Bu \\
	y &=& Cx + Du
\end{eqnarray*}

Das behandelte $IT_1-Glied$ besitzt mit den Anfangswerten \[x_1(0) = y(0),  x_2(0) = \dot y(0)\] folgende Zustandsraumdarstellung:

\begin{eqnarray*}
	\dot x &=& \left(\begin{array}{cc} 0 & 1\\ 0 & -\frac{1}{7}\end{array}\right) x + \left(\begin{array}{c} 0\\ \frac{4}{7}\end{array}\right) u \\
	y &=& \left(\begin{array}{cc} 1 & 0\end{array}\right) x
\end{eqnarray*} 




UNTERSCHIEDLICH ZU MATLAB LÖSUNG???
Matlab:

\begin{eqnarray*}
	\dot x &=& \left(\begin{array}{cc} -\frac{1}{7} & 0\\ \frac{1}{4} & 0\end{array}\right) x + \left(\begin{array}{c} 2\\ 0\end{array}\right) u  \\
	y &=& \left(\begin{array}{cc} 0 & \frac{8}{7}\end{array}\right) x
\end{eqnarray*}