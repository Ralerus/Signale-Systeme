\subsection{Die Übertragungsfunktion}

Formel \ref{eq:G(s)_1} zeigt die allgemeine Übertragungsfunktion eines IT1-Gliedes.\\ In Formel \ref{eq:G(s)_2} wird ein konkretes IT1-Glied dargestellt, das in diesem Skript analysiert wird. Formel \ref{eq:G(s)_3} zeigt dessen ausmultiplizierte Form, sodass die Eigenschaften des Systems leichter zu lesen sind.

\begin{eqnarray}
G(s) &=& \frac{Y(s)}{U(s)} \\
\label{eq:G(s)_1}
&=& \frac{K}{(T_I+T_1s)*s} \\
\label{eq:G(s)_2}
&=& \frac{4}{(1+7s)s} \\
\label{eq:G(s)_3}
&=& \frac{4}{7s^2+s}
\end{eqnarray}