\section{Das System}

\subsection{Die Übertragungsfunktion}

Formel \ref{eq:G(s)_1} zeigt die Übertragungsfunktion die in diesem Skript analysiert wird. Formel \ref{eq:G(s)_3} zeigt dann die ausmultiplizierte Formel, sodass die Eigenschaften des Systems leichter zu lesen sind.

\begin{eqnarray}
    \label{eq:G(s)_1}
    G(s) &=& \frac{Y(s)}{U(s)} \\
    \label{eq:G(s)_2}
    &=& \frac{(1-2s)(1-3s)}{(1+4s)(1+5s)(1+6s)} \\
    \label{eq:G(s)_3}
    &=& \frac{6s^2 - 5s + 1}{120s^3 + 74s^2 + 15s +1}
\end{eqnarray}

\subsection{Eigenschaften des Systems}

Folgend eine Liste der Eigenschaften des Systems, die sich aus der Übertragungsfunktion $G(s)$ ergibt:

\begin{itemize}
    \item Nennergrad - Zählergrad = 1, dadruch ist das System nicht sprungfähig, antwortet aber sofort mit einer Steigung. Außerdem ist die $D-Matrix$ (welche einem Sprung am Anfang ensprechen würde) $0$.
    \item Ein $PT_3$-Glieder
    \item P-Verhalten mit positiver Verstärkung
    \item gerade Anzahl nicht-minimalphasige rechtsseitiger Nullstellen: Dadurch antwortet das System erst falsch
    \item es gibt 3 reelle negative Polstellen, somit ist das System stabil
\end{itemize}

\section{Andere Darstellungsformen des Systems}

\subsection{Eingangs-Ausgang Differentialgleichung}

\begin{eqnarray*}
    G(s) =\frac{Y(s)}{U(s)} &=& \frac{6s^2 - 5s + 1}{120s^3 + 74s^2 + 15s +1} \\
    Y(s)(120s^3 + 74s^2 + 15s +1) &=& U(s) (6s^2 - 5s + 1) \\
    120 \dddot y + 74 \ddot y + 15 \dot y &=& 6 \ddot u + 5 \dot u + u
\end{eqnarray*}

\subsection{Zustandsraumdarstellung}

Die allgemeine Form der Zustandsraumdarstellung ist in den folgenden zwei Zeilen zu sehen.

\begin{eqnarray*}
    \dot x &=& Ax + Bu \nonumber \\
    y &=& Cx
\end{eqnarray*}

\subsubsection{originale Zustandsraumdarstellung}

Dabei ist eine Möglichkeit Variante in der Zustandsraumdarstellung das zu behandelnde System zu beschreiben die Folgende:

\begin{eqnarray*}
    \dot x &=& \left(\begin{array}{ccc} -0.6167 & -0.25 & -0.1333\\ 0.5 & 0 & 0\\ 0 & 0.125 & 0 \end{array}\right) x + \left(\begin{array}{c} 0.5\\ 0\\ 0 \end{array}\right) u  \nonumber \\
    y &=& \left(\begin{array}{ccc} 0.1 & -0.1667 & 0.2667 \end{array}\right) x
\end{eqnarray*}

\subsubsection{Ähnlichkeitstransformation}

Durch eine Ähnlichkeitstransformation können mit der Hilfmatrix $T$ in \ref{eq:T} folgenden Matrizen in der Zustandsraumdarstellung erzeugt werden.

\begin{equation*}
    \label{eq:T}
    T = \left(\begin{array}{ccc} 1 & 2 & 3\\ 0 & 1 & 3\\ 9 & 0 & 1 \end{array}\right)
\end{equation*} 

\begin{eqnarray*}
    \dot x &=& \left(\begin{array}{ccc} 0.1771 & -0.2292 & 0.02292\\ 0.3795 & -0.3839 & 0.01339\\ -1.862 & 1.598 & -0.4098 \end{array}\right)x + \left(\begin{array}{c} 0.5\\ 0\\ 4.5 \end{array}\right)u \\
    y &=& \left(\begin{array}{ccc} -0.2429 & 0.319 & 0.0381 \end{array}\right)x
\end{eqnarray*}



\section{Simulation des Systems}

\subsection{Sprungantwort}

In Abbildung \ref{fig:sprung} ist die Reaktion des Systems auf den Heaviside-Funktion zu sehen. Diese ist wie folgt definiert:

\[
\sigma (t) = \begin{cases} 0 & \text{für t < 0} \\ 1 & \text{für t > 0} \end{cases}  
\]

\begin{figure}[H]
    \label{fig:sprung}
    \centering
    \includegraphics[width=0.8\textwidth]{Bilder/Sprungantwort.eps}
    \caption{Sprungantwort}
 \end{figure}


\subsection{Impulsantwort}

In Abbildung \ref{fig:impuls} ist die Reaktion des Systems auf den Dirac-Funktion zu sehen. Die Dirac-Funktion ist die Ableitung der Heaviside-Funktion und definiert durch:

\[
    \delta (t) \stackrel{D'}{=} \frac{d}{dt} \sigma (t)
\]

Die wichtigste Eigenschaft der Dirac-Funktion ist, dass die Fläche unter dem Impuls exakt $1$ ist. Daraus folgt:

\[
    \int_{-\infty}^\infty \delta (t) dx = 1
\]

\begin{figure}[H]
    \label{fig:impuls}
    \centering
    \includegraphics[width=0.8\textwidth]{Bilder/ImpulsAntwort.eps}
    \caption{Impulsantwort}
\end{figure}

\subsection{explizite Lösung}

Durch die inverse Laplacetransformation lässt sich die explizite Lösung des Systems für die Impulsfunktion als Eingangssignal ermitteln.

\[
    L^{-1}\{\frac{6s^2 - 5s + 1}{120s^3 + 74s^2 + 15s +1}\} = \frac{21\,{\mathrm{e}}^{-\frac{t}{4}}}{4}-\frac{56\,{\mathrm{e}}^{-\frac{t}{5}}}{5}+6\,{\mathrm{e}}^{-\frac{t}{6}}
\]


\subsection{statischer Verstärkungsfaktor}

Wie in Abbildung \ref{fig:sprung} zu sehen pendelt sich der Ausgang des System bei $1$ ein. Das passiert, weil $ k = 1$ ist.
Ein Graph des Verstärkungsfaktor, wobei auf der $x-Achse$ die konstanten Eingänge und auf der $y-Achse$ die Ausgänge des Systems für $t \rightarrow \infty$.

\begin{figure}[H]
    \label{fig:Kennline}
    \centering
    \includegraphics[width=0.5\textwidth]{Bilder/Kennlinie.eps}
    \caption{statischer Verstärkungsfaktor}
\end{figure}


\section{Verhalten bei Schwingungseingängen}

\subsection{Nullstellen und Pole}

Abbildung \ref{fig:pzmap} zeigt einen Pole-Zero-Plot zu dem System, welcher als \textbf{x} markiert die Nullstellen des Systems und mit \textbf{o} markiert die Polstellen des System zeigt.

Das System hat 3 Polstellen auf der reellen Achse in der linken Halbebene, was ebenfalls die Stabilität des Systems zeigt, und 2 nicht-minimalphasige Nullstellen auf der reellen Achse, also in der rechten Halbebene.

\begin{figure}[H]
    \label{fig:pzmap}
    \centering
    \includegraphics[width=0.8\textwidth]{Bilder/PoleZeroPlot.eps}
    \caption{Pole-Zero-Plot}
 \end{figure}


\subsection{Bode-Plot}

In Abbildung \ref{fig:bode} ist die Amplitude und die Phase des Systemausgangs in Abhängigkeit zur Frequenz (logarithmisch skaliert).

\begin{figure}[H]
    \label{fig:bode}
    \centering
    \includegraphics[width=0.8\textwidth]{Bilder/Bode.eps}
    \caption{Bode-Plot}
 \end{figure}

\subsection{Nyquist-Plot}

Die Ortskurve oder auch Nyquist-Plot in \ref{fig:nyquist} ist ein Graph der die Amplitude und Phase des Systems für Schwingungen mit allen Frequenzen darstellt.

\begin{figure}[H]
    \label{fig:nyquist}
    \centering
    \includegraphics[width=0.8\textwidth]{Bilder/Nyquist.eps}
    \caption{Ortskurve / Nyquist-Plot}
 \end{figure}