\section{Explizite Lösung}

Durch die inverse Laplacetransformation lässt sich die explizite Lösung des Systems für die Impulsfunktion als Eingangssignal ermitteln:

\begin{eqnarray*}
	g(t) &=& L^{-1}\{G(s)\}  \\ 
	&=& L^{-1}\{\frac{4}{7s^2 + s}\}\\
	&=& 4 - 4*e^{-\frac{t}{7}}
\end{eqnarray*}

\noindent
Durch die Integration der expliziten Lösung kommen wir auf h(t):

\begin{eqnarray*}
	h(t) &=& \int_{0}^{t}g(\tau)d\tau \\ 
	&=& 4 * t + 28*e^{-\frac{t}{7}}
\end{eqnarray*}

\noindent
Die explizite Lösung des Zustandsraum lässt sich mit der Systemmatrix A, der Eingangsmatrix b, der Ausgangsmatrix c sowie der Durchgangsmatrix D des Zustandsraumes über folgende Formeln berechnen:
\begin{eqnarray*}
 \underline{x}(t) &=& e^{A(t-t_{0})} \underline{x}(t_0) + \int_{t_0}^{t}e^{A(t-\tau)}  \underline{b}u(\tau) d\tau \\
 y &=& c\underline{x}(t) + Du(t)
\end{eqnarray*}

\noindent
Für die homogene Lösung, die dem Eigenvorgang des Systems entspricht, ergibt sich folglich für \(t_0=0\): 
\begin{eqnarray*}
\underline{x}(t) &=& e^{At} \underline{x}(0) \\
 \left[\begin{array}{c} x_1 \\ x_2 \end{array}\right]&=& \left[\begin{array}{cc} 1 & 7 - 7e^{-\frac{1}{7}t} \\ 0 & e^{-\frac{1}{7}t} \end{array}\right]  \left[\begin{array}{c}x_1(0) \\ x_2(0)\end{array}\right] \\ \\
 x_1(t) &=& x_1(0) + 7x_2(0) - 7e^{-\frac{1}{7}t}x_2(0) \\ 
 x_2(t) &=& e^{-\frac{1}{7}t}x_2(0) \\
\end{eqnarray*}


\noindent
Die erste Zeile zeigt die explizite Lösung von \(x_1 = y\), die zweite Zeile die explizite Lösung von \(x_2 = \dot{y} = \dot{x_1}\).
